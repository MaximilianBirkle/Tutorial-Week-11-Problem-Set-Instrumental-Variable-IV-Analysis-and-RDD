% Options for packages loaded elsewhere
\PassOptionsToPackage{unicode}{hyperref}
\PassOptionsToPackage{hyphens}{url}
\documentclass[
]{article}
\usepackage{xcolor}
\usepackage[margin=1in]{geometry}
\usepackage{amsmath,amssymb}
\setcounter{secnumdepth}{5}
\usepackage{iftex}
\ifPDFTeX
  \usepackage[T1]{fontenc}
  \usepackage[utf8]{inputenc}
  \usepackage{textcomp} % provide euro and other symbols
\else % if luatex or xetex
  \usepackage{unicode-math} % this also loads fontspec
  \defaultfontfeatures{Scale=MatchLowercase}
  \defaultfontfeatures[\rmfamily]{Ligatures=TeX,Scale=1}
\fi
\usepackage{lmodern}
\ifPDFTeX\else
  % xetex/luatex font selection
\fi
% Use upquote if available, for straight quotes in verbatim environments
\IfFileExists{upquote.sty}{\usepackage{upquote}}{}
\IfFileExists{microtype.sty}{% use microtype if available
  \usepackage[]{microtype}
  \UseMicrotypeSet[protrusion]{basicmath} % disable protrusion for tt fonts
}{}
\makeatletter
\@ifundefined{KOMAClassName}{% if non-KOMA class
  \IfFileExists{parskip.sty}{%
    \usepackage{parskip}
  }{% else
    \setlength{\parindent}{0pt}
    \setlength{\parskip}{6pt plus 2pt minus 1pt}}
}{% if KOMA class
  \KOMAoptions{parskip=half}}
\makeatother
\usepackage{color}
\usepackage{fancyvrb}
\newcommand{\VerbBar}{|}
\newcommand{\VERB}{\Verb[commandchars=\\\{\}]}
\DefineVerbatimEnvironment{Highlighting}{Verbatim}{commandchars=\\\{\}}
% Add ',fontsize=\small' for more characters per line
\usepackage{framed}
\definecolor{shadecolor}{RGB}{248,248,248}
\newenvironment{Shaded}{\begin{snugshade}}{\end{snugshade}}
\newcommand{\AlertTok}[1]{\textcolor[rgb]{0.94,0.16,0.16}{#1}}
\newcommand{\AnnotationTok}[1]{\textcolor[rgb]{0.56,0.35,0.01}{\textbf{\textit{#1}}}}
\newcommand{\AttributeTok}[1]{\textcolor[rgb]{0.13,0.29,0.53}{#1}}
\newcommand{\BaseNTok}[1]{\textcolor[rgb]{0.00,0.00,0.81}{#1}}
\newcommand{\BuiltInTok}[1]{#1}
\newcommand{\CharTok}[1]{\textcolor[rgb]{0.31,0.60,0.02}{#1}}
\newcommand{\CommentTok}[1]{\textcolor[rgb]{0.56,0.35,0.01}{\textit{#1}}}
\newcommand{\CommentVarTok}[1]{\textcolor[rgb]{0.56,0.35,0.01}{\textbf{\textit{#1}}}}
\newcommand{\ConstantTok}[1]{\textcolor[rgb]{0.56,0.35,0.01}{#1}}
\newcommand{\ControlFlowTok}[1]{\textcolor[rgb]{0.13,0.29,0.53}{\textbf{#1}}}
\newcommand{\DataTypeTok}[1]{\textcolor[rgb]{0.13,0.29,0.53}{#1}}
\newcommand{\DecValTok}[1]{\textcolor[rgb]{0.00,0.00,0.81}{#1}}
\newcommand{\DocumentationTok}[1]{\textcolor[rgb]{0.56,0.35,0.01}{\textbf{\textit{#1}}}}
\newcommand{\ErrorTok}[1]{\textcolor[rgb]{0.64,0.00,0.00}{\textbf{#1}}}
\newcommand{\ExtensionTok}[1]{#1}
\newcommand{\FloatTok}[1]{\textcolor[rgb]{0.00,0.00,0.81}{#1}}
\newcommand{\FunctionTok}[1]{\textcolor[rgb]{0.13,0.29,0.53}{\textbf{#1}}}
\newcommand{\ImportTok}[1]{#1}
\newcommand{\InformationTok}[1]{\textcolor[rgb]{0.56,0.35,0.01}{\textbf{\textit{#1}}}}
\newcommand{\KeywordTok}[1]{\textcolor[rgb]{0.13,0.29,0.53}{\textbf{#1}}}
\newcommand{\NormalTok}[1]{#1}
\newcommand{\OperatorTok}[1]{\textcolor[rgb]{0.81,0.36,0.00}{\textbf{#1}}}
\newcommand{\OtherTok}[1]{\textcolor[rgb]{0.56,0.35,0.01}{#1}}
\newcommand{\PreprocessorTok}[1]{\textcolor[rgb]{0.56,0.35,0.01}{\textit{#1}}}
\newcommand{\RegionMarkerTok}[1]{#1}
\newcommand{\SpecialCharTok}[1]{\textcolor[rgb]{0.81,0.36,0.00}{\textbf{#1}}}
\newcommand{\SpecialStringTok}[1]{\textcolor[rgb]{0.31,0.60,0.02}{#1}}
\newcommand{\StringTok}[1]{\textcolor[rgb]{0.31,0.60,0.02}{#1}}
\newcommand{\VariableTok}[1]{\textcolor[rgb]{0.00,0.00,0.00}{#1}}
\newcommand{\VerbatimStringTok}[1]{\textcolor[rgb]{0.31,0.60,0.02}{#1}}
\newcommand{\WarningTok}[1]{\textcolor[rgb]{0.56,0.35,0.01}{\textbf{\textit{#1}}}}
\usepackage{graphicx}
\makeatletter
\newsavebox\pandoc@box
\newcommand*\pandocbounded[1]{% scales image to fit in text height/width
  \sbox\pandoc@box{#1}%
  \Gscale@div\@tempa{\textheight}{\dimexpr\ht\pandoc@box+\dp\pandoc@box\relax}%
  \Gscale@div\@tempb{\linewidth}{\wd\pandoc@box}%
  \ifdim\@tempb\p@<\@tempa\p@\let\@tempa\@tempb\fi% select the smaller of both
  \ifdim\@tempa\p@<\p@\scalebox{\@tempa}{\usebox\pandoc@box}%
  \else\usebox{\pandoc@box}%
  \fi%
}
% Set default figure placement to htbp
\def\fps@figure{htbp}
\makeatother
\setlength{\emergencystretch}{3em} % prevent overfull lines
\providecommand{\tightlist}{%
  \setlength{\itemsep}{0pt}\setlength{\parskip}{0pt}}
\usepackage{booktabs}
\usepackage{longtable}
\usepackage{array}
\usepackage{multirow}
\usepackage{wrapfig}
\usepackage{float}
\usepackage{colortbl}
\usepackage{pdflscape}
\usepackage{tabu}
\usepackage{threeparttable}
\usepackage{threeparttablex}
\usepackage[normalem]{ulem}
\usepackage{makecell}
\usepackage{xcolor}
\usepackage{bookmark}
\IfFileExists{xurl.sty}{\usepackage{xurl}}{} % add URL line breaks if available
\urlstyle{same}
\hypersetup{
  pdftitle={Tutorial Week 11: Instrumental Variable (IV) Analysis and RDD},
  pdfauthor={Maximilian Birkle; Daniel Lehmann; Henry Lucas},
  hidelinks,
  pdfcreator={LaTeX via pandoc}}

\title{Tutorial Week 11: Instrumental Variable (IV) Analysis and RDD}
\author{Maximilian Birkle \and Daniel Lehmann \and Henry Lucas}
\date{2025-11-19}

\begin{document}
\maketitle

{
\setcounter{tocdepth}{2}
\tableofcontents
}
\section{Setup and Package Loading}\label{setup-and-package-loading}

\begin{Shaded}
\begin{Highlighting}[]
\CommentTok{\# Install packages if needed}
\NormalTok{packages }\OtherTok{\textless{}{-}} \FunctionTok{c}\NormalTok{(}\StringTok{"AER"}\NormalTok{, }\StringTok{"haven"}\NormalTok{, }\StringTok{"dplyr"}\NormalTok{, }\StringTok{"ggplot2"}\NormalTok{, }\StringTok{"stargazer"}\NormalTok{,}
              \StringTok{"boot"}\NormalTok{, }\StringTok{"lmtest"}\NormalTok{, }\StringTok{"sandwich"}\NormalTok{, }\StringTok{"knitr"}\NormalTok{, }\StringTok{"kableExtra"}\NormalTok{,}
              \StringTok{"rdrobust"}\NormalTok{, }\StringTok{"rddensity"}\NormalTok{, }\StringTok{"rdlocrand"}\NormalTok{)}

\ControlFlowTok{for}\NormalTok{ (pkg }\ControlFlowTok{in}\NormalTok{ packages) \{}
  \ControlFlowTok{if}\NormalTok{ (}\SpecialCharTok{!}\FunctionTok{require}\NormalTok{(pkg, }\AttributeTok{character.only =} \ConstantTok{TRUE}\NormalTok{)) \{}
    \FunctionTok{install.packages}\NormalTok{(pkg)}
    \FunctionTok{library}\NormalTok{(pkg, }\AttributeTok{character.only =} \ConstantTok{TRUE}\NormalTok{)}
\NormalTok{  \}}
\NormalTok{\}}
\end{Highlighting}
\end{Shaded}

\section{Load Data}\label{load-data}

\begin{Shaded}
\begin{Highlighting}[]
\CommentTok{\# Load the AJR (2001) replication data}
\NormalTok{ajr\_data }\OtherTok{\textless{}{-}}\NormalTok{ haven}\SpecialCharTok{::}\FunctionTok{read\_dta}\NormalTok{(}\StringTok{"acemoglu.dta"}\NormalTok{)}
\FunctionTok{cat}\NormalTok{(}\StringTok{"Observations:"}\NormalTok{, }\FunctionTok{nrow}\NormalTok{(ajr\_data), }\StringTok{"}\SpecialCharTok{\textbackslash{}n}\StringTok{"}\NormalTok{)}
\end{Highlighting}
\end{Shaded}

\begin{verbatim}
## Observations: 163
\end{verbatim}

\begin{Shaded}
\begin{Highlighting}[]
\FunctionTok{cat}\NormalTok{(}\StringTok{"Variables:"}\NormalTok{, }\FunctionTok{ncol}\NormalTok{(ajr\_data), }\StringTok{"}\SpecialCharTok{\textbackslash{}n}\StringTok{"}\NormalTok{)}
\end{Highlighting}
\end{Shaded}

\begin{verbatim}
## Variables: 12
\end{verbatim}

\section{Problem 1: Instrumental Variable
Analysis}\label{problem-1-instrumental-variable-analysis}

\subsection{Q1. Writing and Estimating the IV Model (10
pts)}\label{q1.-writing-and-estimating-the-iv-model-10-pts}

\begin{enumerate}
\def\labelenumi{\arabic{enumi}.}
\item
  \textbf{Write down the two-equation IV system} (first stage and second
  stage).

  Define explicitly:

  \begin{itemize}
  \tightlist
  \item
    \(Y_i\): outcome
  \item
    \(T_i\): endogenous regressor
  \item
    \(Z_i\): instrument
  \item
    \(X_i\): controls
  \end{itemize}
\item
  Estimate the following:

  \begin{itemize}
  \item
    \textbf{First stage:}

    \[T_i = \alpha_1 + \beta_1 Z_i + \mathbf{X}_i' \gamma_1 + \varepsilon_{1i}\]
  \item
    \textbf{Reduced form:}

    \[Y_i = \alpha_2 + \delta_1 Z_i + \mathbf{X}_i' \gamma_2 + \varepsilon_{2i}\]
  \item
    \textbf{2SLS:}

    \[Y_i = \alpha_3 + \beta_2 T_i + \mathbf{X}_i' \gamma_3 + \varepsilon_{3i}\]
  \end{itemize}
\item
  Report and interpret:

  \begin{itemize}
  \tightlist
  \item
    The reduced-form coefficient on \(Z_i\).
  \item
    The first-stage coefficient on \(Z_i\).
  \item
    The \textbf{first-stage F-statistic}. Is it above the rule-of-thumb
    threshold of \(\approx\) 10?
  \item
    The 2SLS estimate of the effect of \(T_i\) on \(Y_i\). Is it
    statistically significant?
  \end{itemize}
\end{enumerate}

\begin{Shaded}
\begin{Highlighting}[]
\CommentTok{\# We define our control variables as a string for an easier formula building}
\NormalTok{controls }\OtherTok{\textless{}{-}} \StringTok{"lat\_abst + f\_french + f\_brit + sjlofr + catho80 + muslim80 + no\_cpm80"}

\CommentTok{\# First Stage: (Z\_i {-}\textgreater{} T\_i): We regress Endogeneous (avexpr) on Instrument (logem4) + Controls}
\NormalTok{fs\_formula }\OtherTok{\textless{}{-}} \FunctionTok{as.formula}\NormalTok{(}\FunctionTok{paste}\NormalTok{(}\StringTok{"avexpr \textasciitilde{} logem4 +"}\NormalTok{, controls))}
\NormalTok{fs\_model }\OtherTok{\textless{}{-}} \FunctionTok{lm}\NormalTok{(fs\_formula, }\AttributeTok{data =}\NormalTok{ ajr\_data)}

\CommentTok{\# Reduced Form: (Z\_i {-}\textgreater{} Y\_i): We regress Outcome (logpgp95) on Instrument (logem4) + Controls}
\NormalTok{rf\_formula }\OtherTok{\textless{}{-}} \FunctionTok{as.formula}\NormalTok{(}\FunctionTok{paste}\NormalTok{(}\StringTok{"logpgp95 \textasciitilde{} logem4 +"}\NormalTok{, controls))}
\NormalTok{rf\_model }\OtherTok{\textless{}{-}} \FunctionTok{lm}\NormalTok{(rf\_formula, }\AttributeTok{data =}\NormalTok{ ajr\_data)}

\CommentTok{\# 2SLS: (T\_i {-}\textgreater{} Y\_i): We regress Outcome (logpgp95) on Endogeneous (avexpr) + Controls}
\CommentTok{\# For this stage, we have to use the ivreg{-}command. Since T\_i is endogeneous (Y {-}\textgreater{} X),}
\CommentTok{\# we need to run this command in order to isolate the specific variation in institutions}
\CommentTok{\# caused by historical mortality rates and uses only that chunk of variation to explain GDP}
\NormalTok{iv\_formula }\OtherTok{\textless{}{-}} \FunctionTok{as.formula}\NormalTok{(}\FunctionTok{paste}\NormalTok{(}\StringTok{"logpgp95 \textasciitilde{} avexpr +"}\NormalTok{, controls, }\StringTok{"| logem4 +"}\NormalTok{, controls))}
\NormalTok{iv\_model }\OtherTok{\textless{}{-}} \FunctionTok{ivreg}\NormalTok{(iv\_formula, }\AttributeTok{data =}\NormalTok{ ajr\_data)}

\CommentTok{\# For an instrumental variable to work, we have to verify that our instrumental variable Z\_i}
\CommentTok{\# is not a \textquotesingle{}weak instrument\textquotesingle{}. Therefore, the instrument (Mortality, Z\_i) must trigger a }
\CommentTok{\# significant change in the endogeneous variable (Institutions, T\_i)}
\NormalTok{f\_test }\OtherTok{\textless{}{-}} \FunctionTok{linearHypothesis}\NormalTok{(fs\_model, }\StringTok{"logem4 = 0"}\NormalTok{)}
\NormalTok{f\_stat }\OtherTok{\textless{}{-}}\NormalTok{ f\_test}\SpecialCharTok{$}\NormalTok{F[}\DecValTok{2}\NormalTok{]}

\CommentTok{\# Output results using stargazer}
\FunctionTok{stargazer}\NormalTok{(fs\_model, rf\_model, iv\_model, }
          \AttributeTok{type =} \StringTok{"latex"}\NormalTok{, }
          \AttributeTok{title =} \StringTok{"IV Analysis Results"}\NormalTok{,}
          \AttributeTok{column.labels =} \FunctionTok{c}\NormalTok{(}\StringTok{"First Stage"}\NormalTok{, }\StringTok{"Reduced Form"}\NormalTok{, }\StringTok{"2SLS"}\NormalTok{),}
          \AttributeTok{dep.var.labels =} \FunctionTok{c}\NormalTok{(}\StringTok{"Exprop. Risk (T)"}\NormalTok{, }\StringTok{"Log GDP (Y)"}\NormalTok{, }\StringTok{"Log GDP (Y)"}\NormalTok{),}
          \AttributeTok{covariate.labels =} \FunctionTok{c}\NormalTok{(}\StringTok{"Log Settler Mortality (Z)"}\NormalTok{, }\StringTok{"Exprop. Risk (T)"}\NormalTok{),}
          \AttributeTok{add.lines =} \FunctionTok{list}\NormalTok{(}\FunctionTok{c}\NormalTok{(}\StringTok{"First{-}Stage F{-}stat"}\NormalTok{, }\FunctionTok{round}\NormalTok{(f\_stat, }\DecValTok{2}\NormalTok{), }\StringTok{"{-}"}\NormalTok{, }\StringTok{"{-}"}\NormalTok{)))}
\end{Highlighting}
\end{Shaded}

\% Table created by stargazer v.5.2.3 by Marek Hlavac, Social Policy
Institute. E-mail: marek.hlavac at gmail.com \% Date and time: Mi, Nov
19, 2025 - 13:53:21

\begin{table}[!htbp] \centering 
  \caption{IV Analysis Results} 
  \label{} 
\begin{tabular}{@{\extracolsep{5pt}}lccc} 
\\[-1.8ex]\hline 
\hline \\[-1.8ex] 
 & \multicolumn{3}{c}{\textit{Dependent variable:}} \\ 
\cline{2-4} 
\\[-1.8ex] & Exprop. Risk (T) & \multicolumn{2}{c}{Log GDP (Y)} \\ 
\\[-1.8ex] & \textit{OLS} & \textit{OLS} & \textit{instrumental} \\ 
 & \textit{} & \textit{} & \textit{variable} \\ 
 & First Stage & Reduced Form & 2SLS \\ 
\\[-1.8ex] & (1) & (2) & (3)\\ 
\hline \\[-1.8ex] 
 Log Settler Mortality (Z) & $-$0.385$^{**}$ & $-$0.386$^{***}$ &  \\ 
  & (0.152) & (0.083) &  \\ 
  & & & \\ 
 Exprop. Risk (T) &  &  & 1.098$^{***}$ \\ 
  &  &  & (0.338) \\ 
  & & & \\ 
 lat\_abst & 3.277$^{**}$ & 2.039$^{***}$ & $-$1.841 \\ 
  & (1.282) & (0.706) & (1.906) \\ 
  & & & \\ 
 f\_french & 0.140 & 0.131 & 0.279 \\ 
  & (0.504) & (0.272) & (0.422) \\ 
  & & & \\ 
 f\_brit & 0.060 & 0.186 & 0.231 \\ 
  & (0.530) & (0.285) & (0.439) \\ 
  & & & \\ 
 sjlofr & $-$0.618 & $-$0.184 & 0.527 \\ 
  & (0.651) & (0.339) & (0.551) \\ 
  & & & \\ 
 catho80 & 0.001 & 0.005 & $-$0.001 \\ 
  & (0.013) & (0.007) & (0.011) \\ 
  & & & \\ 
 muslim80 & $-$0.005 & $-$0.006 & $-$0.008 \\ 
  & (0.012) & (0.006) & (0.010) \\ 
  & & & \\ 
 no\_cpm80 & 0.004 & $-$0.002 & $-$0.016 \\ 
  & (0.013) & (0.007) & (0.012) \\ 
  & & & \\ 
 Constant & 7.959$^{***}$ & 9.487$^{***}$ & 1.435 \\ 
  & (1.545) & (0.814) & (2.120) \\ 
  & & & \\ 
\hline \\[-1.8ex] 
First-Stage F-stat & 6.42 & - & - \\ 
Observations & 74 & 81 & 70 \\ 
R$^{2}$ & 0.424 & 0.601 & 0.218 \\ 
Adjusted R$^{2}$ & 0.353 & 0.556 & 0.116 \\ 
Residual Std. Error & 1.265 (df = 65) & 0.706 (df = 72) & 0.996 (df = 61) \\ 
F Statistic & 5.985$^{***}$ (df = 8; 65) & 13.532$^{***}$ (df = 8; 72) &  \\ 
\hline 
\hline \\[-1.8ex] 
\textit{Note:}  & \multicolumn{3}{r}{$^{*}$p$<$0.1; $^{**}$p$<$0.05; $^{***}$p$<$0.01} \\ 
\end{tabular} 
\end{table}

\subsection{Q2. Randomization and Resampling (10
pts)}\label{q2.-randomization-and-resampling-10-pts}

\subsubsection{Q2(a). Permutation Test}\label{q2a.-permutation-test}

\begin{enumerate}
\def\labelenumi{\arabic{enumi}.}
\item
  \textbf{State the null hypothesis} being tested.
\item
  Conduct a \textbf{permutation test} for the IV coefficient:

  \begin{itemize}
  \tightlist
  \item
    Shuffle the endogenous variable (or fitted values) while holding
    other variables fixed.
  \item
    Re-estimate the IV model for each permutation.
  \item
    Construct the empirical null distribution.
  \item
    Report the \textbf{permutation p-value}.
  \end{itemize}
\item
  Compare this p-value to the \textbf{normal-approximation p-value} from
  your 2SLS output.

  Discuss any differences and what they imply for small-sample IV
  inference.
\end{enumerate}

\subsubsection{Q2(b). Bootstrap Confidence
Intervals}\label{q2b.-bootstrap-confidence-intervals}

Using bootstrap resampling of observations:

\begin{enumerate}
\def\labelenumi{\arabic{enumi}.}
\item
  Generate bootstrap 2SLS estimates of the coefficient on \(T_i\).
\item
  Construct three 95\% confidence intervals:

  \begin{itemize}
  \tightlist
  \item
    \textbf{Efron percentile}
  \item
    \textbf{Bias-corrected (BC---google this one.)}
  \end{itemize}
\item
  Compare the three CIs:

  \begin{itemize}
  \tightlist
  \item
    Do they include zero?
  \item
    Are they wider or narrower?
  \item
    What does this imply about the sampling distribution?
  \end{itemize}
\end{enumerate}

\subsubsection{Q2(c). Conceptual: Permutation vs
Bootstrap}\label{q2c.-conceptual-permutation-vs-bootstrap}

Explain---precisely---what the \textbf{permutation test} and the
\textbf{bootstrap} each measure.

What is held fixed? What is resampled?

Why do they answer conceptually different questions?

\subsection{Q3. Instrument Validity and Timing (10
pts)}\label{q3.-instrument-validity-and-timing-10-pts}

\subsubsection{Q3.1 Causal Priority in AJR's
Theory}\label{q3.1-causal-priority-in-ajrs-theory}

\textbf{Why settler mortality must be causally prior to institutions:}

{[}Write your answer here{]}

Key points to address:

\begin{itemize}
\tightlist
\item
  The logic of instrumental variables requires that
  \(Z \rightarrow T \rightarrow Y\) (no reverse causation)
\item
  In AJR's theory, high settler mortality \(\rightarrow\) extractive
  institutions \(\rightarrow\) lower growth
\item
  If institutions could affect mortality rates retroactively, the IV
  assumption fails
\item
  The exclusion restriction requires mortality affects GDP ONLY through
  institutions
\end{itemize}

\subsubsection{Q3.2 Timing Problems}\label{q3.2-timing-problems}

\textbf{Issues with timing of settler mortality measurements:}

{[}Write your answer here{]}

Address each IV assumption:

\textbf{Relevance (First-stage):}

\begin{itemize}
\tightlist
\item
  Why does measurement timing affect the strength of the first stage?
\item
  Consider data availability and measurement error
\end{itemize}

\textbf{Exclusion Restriction:}

\begin{itemize}
\tightlist
\item
  If mortality was measured long after colonization, what other channels
  might exist?
\item
  Could later mortality reflect economic conditions rather than cause
  them?
\end{itemize}

\textbf{Independence:}

\begin{itemize}
\tightlist
\item
  Are there confounders that affect both late-measured mortality and
  outcomes?
\item
  Geographic or climatic factors?
\end{itemize}

\subsubsection{Q3.3 Assessment of AJR
Results}\label{q3.3-assessment-of-ajr-results}

Based on your analysis:

\textbf{Your assessment:}

{[}Write your answer here{]}

Consider:

\begin{itemize}
\tightlist
\item
  Is the first stage strong enough?
\item
  Are the IV assumptions plausible given the timing issues?
\item
  What are the main threats to validity?
\item
  Would you believe the causal interpretation?
\end{itemize}

\subsection{Q4. Albouy's Critique of AJR (10
pts)}\label{q4.-albouys-critique-of-ajr-10-pts}

\subsubsection{Q4(a). Measurement
Problems}\label{q4a.-measurement-problems}

\textbf{Key measurement issues raised by Albouy (2012):}

\begin{enumerate}
\def\labelenumi{\arabic{enumi}.}
\tightlist
\item
\item
\end{enumerate}

\textbf{Why they matter for IV validity:}

{[}Your answer{]}

\subsubsection{Q4(b). Violations of IV
Assumptions}\label{q4b.-violations-of-iv-assumptions}

\textbf{Relevance:}

Albouy's argument:

{[}Your answer{]}

\textbf{Independence:}

Albouy's argument:

{[}Your answer{]}

\textbf{Exclusion Restriction:}

Albouy's argument:

{[}Your answer{]}

\subsubsection{Q4(c). Sensitivity and Data
Corrections}\label{q4c.-sensitivity-and-data-corrections}

After Albouy reconstructs and corrects the mortality data:

\textbf{Effects on:}

\begin{enumerate}
\def\labelenumi{\arabic{enumi}.}
\item
  \textbf{First stage:} {[}Your answer - what happens to F-statistic and
  coefficient?{]}
\item
  \textbf{Reduced form:} {[}Your answer - does the relationship
  weaken?{]}
\item
  \textbf{2SLS estimates:} {[}Your answer - how do the causal estimates
  change?{]}
\end{enumerate}

\textbf{What this reveals about stability:}

{[}Your answer - are the AJR findings robust or fragile?{]}

\subsubsection{Q4(d). Interpretation}\label{q4d.-interpretation}

\textbf{Do you believe the AJR conclusions still hold?}

{[}Your answer{]}

\textbf{Or do the methodological issues undermine the core causal
claim?}

{[}Your answer{]}

\textbf{Justification:}

{[}Provide clear reasoning based on:

\begin{itemize}
\tightlist
\item
  The strength of Albouy's critique
\item
  Your replication results
\item
  The plausibility of IV assumptions
\item
  The sensitivity of findings to data corrections{]}
\end{itemize}

\section{Problem 2: Regression Discontinuity
Design}\label{problem-2-regression-discontinuity-design}

\subsection{Part A - Data Loading and
Setup}\label{part-a---data-loading-and-setup}

\begin{Shaded}
\begin{Highlighting}[]
\CommentTok{\# Ensure RDD packages are loaded}
\FunctionTok{library}\NormalTok{(rdrobust)}
\FunctionTok{library}\NormalTok{(rddensity)}
\FunctionTok{library}\NormalTok{(rdlocrand)}
\FunctionTok{library}\NormalTok{(knitr)}
\FunctionTok{library}\NormalTok{(kableExtra)}

\CommentTok{\# Load RD data}
\NormalTok{data }\OtherTok{\textless{}{-}} \FunctionTok{read.csv}\NormalTok{(}\StringTok{"CTV\_2020\_Sage.csv"}\NormalTok{)}

\CommentTok{\# Define outcome, running variable, and covariates}
\NormalTok{Y }\OtherTok{\textless{}{-}}\NormalTok{ data}\SpecialCharTok{$}\NormalTok{mv\_incpartyfor1}
\NormalTok{X }\OtherTok{\textless{}{-}}\NormalTok{ data}\SpecialCharTok{$}\NormalTok{mv\_incparty}

\NormalTok{covs }\OtherTok{\textless{}{-}}\NormalTok{ data[, }\FunctionTok{c}\NormalTok{(}\StringTok{"pibpc"}\NormalTok{, }\StringTok{"population"}\NormalTok{, }\StringTok{"numpar\_candidates\_eff"}\NormalTok{,}
                 \StringTok{"party\_DEM\_wonlag1\_b1"}\NormalTok{, }\StringTok{"party\_PSDB\_wonlag1\_b1"}\NormalTok{,}
                 \StringTok{"party\_PT\_wonlag1\_b1"}\NormalTok{, }\StringTok{"party\_PMDB\_wonlag1\_b1"}\NormalTok{)]}
\NormalTok{covsnm }\OtherTok{\textless{}{-}} \FunctionTok{c}\NormalTok{(}\StringTok{"GDP per capita"}\NormalTok{, }\StringTok{"Population"}\NormalTok{, }\StringTok{"No. Effective Parties"}\NormalTok{,}
            \StringTok{"DEM Victory t{-}1"}\NormalTok{, }\StringTok{"PSDB Victory t{-}1"}\NormalTok{, }\StringTok{"PT Victory t{-}1"}\NormalTok{, }\StringTok{"PMDB Victory t{-}1"}\NormalTok{)}

\FunctionTok{cat}\NormalTok{(}\StringTok{"RDD data loaded successfully!}\SpecialCharTok{\textbackslash{}n}\StringTok{"}\NormalTok{)}
\end{Highlighting}
\end{Shaded}

\begin{verbatim}
## RDD data loaded successfully!
\end{verbatim}

\begin{Shaded}
\begin{Highlighting}[]
\FunctionTok{cat}\NormalTok{(}\StringTok{"Observations:"}\NormalTok{, }\FunctionTok{nrow}\NormalTok{(data), }\StringTok{"}\SpecialCharTok{\textbackslash{}n}\StringTok{"}\NormalTok{)}
\end{Highlighting}
\end{Shaded}

\begin{verbatim}
## Observations: 27455
\end{verbatim}

\begin{Shaded}
\begin{Highlighting}[]
\FunctionTok{cat}\NormalTok{(}\StringTok{"Running variable (X): Incumbent Party\textquotesingle{}s Margin of Victory}\SpecialCharTok{\textbackslash{}n}\StringTok{"}\NormalTok{)}
\end{Highlighting}
\end{Shaded}

\begin{verbatim}
## Running variable (X): Incumbent Party's Margin of Victory
\end{verbatim}

\begin{Shaded}
\begin{Highlighting}[]
\FunctionTok{cat}\NormalTok{(}\StringTok{"Outcome variable (Y): Incumbent Party Victory at t+1}\SpecialCharTok{\textbackslash{}n}\StringTok{"}\NormalTok{)}
\end{Highlighting}
\end{Shaded}

\begin{verbatim}
## Outcome variable (Y): Incumbent Party Victory at t+1
\end{verbatim}

\subsection{Part B - Falsification
Analysis}\label{part-b---falsification-analysis}

\subsubsection{Density Test}\label{density-test}

\begin{Shaded}
\begin{Highlighting}[]
\CommentTok{\# McCrary density test using rddensity}
\NormalTok{rddens }\OtherTok{\textless{}{-}} \FunctionTok{rddensity}\NormalTok{(X)}
\FunctionTok{summary}\NormalTok{(rddens)}
\end{Highlighting}
\end{Shaded}

\begin{verbatim}
## 
## Manipulation testing using local polynomial density estimation.
## 
## Number of obs =       13308
## Model =               unrestricted
## Kernel =              triangular
## BW method =           estimated
## VCE method =          jackknife
## 
## c = 0                 Left of c           Right of c          
## Number of obs         6088                7220                
## Eff. Number of obs    3852                3590                
## Order est. (p)        2                   2                   
## Order bias (q)        3                   3                   
## BW est. (h)           15.493              13.392              
## 
## Method                T                   P > |T|             
## Robust                -0.0757             0.9397
\end{verbatim}

\begin{verbatim}
## 
## P-values of binomial tests (H0: p=0.5).
## 
## Window Length / 2          <c     >=c    P>|T|
## 0.098                      20      26    0.4614
## 0.195                      53      50    0.8439
## 0.293                      80      77    0.8732
## 0.390                     112     114    0.9470
## 0.488                     141     140    1.0000
## 0.585                     180     184    0.8751
## 0.683                     199     213    0.5219
## 0.780                     231     240    0.7125
## 0.878                     254     267    0.5991
## 0.975                     278     296    0.4780
\end{verbatim}

\begin{Shaded}
\begin{Highlighting}[]
\CommentTok{\# Plot the density}
\FunctionTok{rdplotdensity}\NormalTok{(rddens, }\AttributeTok{X =}\NormalTok{ data}\SpecialCharTok{$}\NormalTok{mv\_incparty[}\SpecialCharTok{!}\FunctionTok{is.na}\NormalTok{(data}\SpecialCharTok{$}\NormalTok{mv\_incparty)],}
              \AttributeTok{xlab =} \StringTok{"Incumbent Party\textquotesingle{}s Margin of Victory at t"}\NormalTok{,}
              \AttributeTok{ylab =} \StringTok{"Estimated density"}\NormalTok{)}
\end{Highlighting}
\end{Shaded}

\pandocbounded{\includegraphics[keepaspectratio]{Tutorial-Week-11-IV-RDD-Analysis_files/figure-latex/density-test-1.pdf}}

\begin{verbatim}
## $Estl
## Call: lpdensity
## 
## Sample size                                      6088
## Polynomial order for point estimation    (p=)    2
## Order of derivative estimated            (v=)    1
## Polynomial order for confidence interval (q=)    3
## Kernel function                                  triangular
## Scaling factor                                   0.457503569549861
## Bandwidth method                                 user provided
## 
## Use summary(...) to show estimates.
## 
## $Estr
## Call: lpdensity
## 
## Sample size                                      7220
## Polynomial order for point estimation    (p=)    2
## Order of derivative estimated            (v=)    1
## Polynomial order for confidence interval (q=)    3
## Kernel function                                  triangular
## Scaling factor                                   0.542571578868265
## Bandwidth method                                 user provided
## 
## Use summary(...) to show estimates.
## 
## $Estplot
\end{verbatim}

\pandocbounded{\includegraphics[keepaspectratio]{Tutorial-Week-11-IV-RDD-Analysis_files/figure-latex/density-test-2.pdf}}

\begin{Shaded}
\begin{Highlighting}[]
\CommentTok{\# Create summary table for density test}
\NormalTok{density\_results }\OtherTok{\textless{}{-}} \FunctionTok{data.frame}\NormalTok{(}
  \AttributeTok{Test =} \StringTok{"Manipulation Test"}\NormalTok{,}
  \AttributeTok{Statistic =} \FunctionTok{sprintf}\NormalTok{(}\StringTok{"\%.4f"}\NormalTok{, rddens}\SpecialCharTok{$}\NormalTok{test}\SpecialCharTok{$}\NormalTok{t\_jk),}
  \StringTok{\textasciigrave{}}\AttributeTok{p{-}value}\StringTok{\textasciigrave{}} \OtherTok{=} \FunctionTok{sprintf}\NormalTok{(}\StringTok{"\%.4f"}\NormalTok{, rddens}\SpecialCharTok{$}\NormalTok{test}\SpecialCharTok{$}\NormalTok{p\_jk),}
  \AttributeTok{Bandwidth =} \FunctionTok{sprintf}\NormalTok{(}\StringTok{"\%.2f / \%.2f"}\NormalTok{, rddens}\SpecialCharTok{$}\NormalTok{h}\SpecialCharTok{$}\NormalTok{left, rddens}\SpecialCharTok{$}\NormalTok{h}\SpecialCharTok{$}\NormalTok{right),}
  \StringTok{\textasciigrave{}}\AttributeTok{N (Left)}\StringTok{\textasciigrave{}} \OtherTok{=} \FunctionTok{format}\NormalTok{(rddens}\SpecialCharTok{$}\NormalTok{N}\SpecialCharTok{$}\NormalTok{left, }\AttributeTok{big.mark =} \StringTok{","}\NormalTok{),}
  \StringTok{\textasciigrave{}}\AttributeTok{N (Right)}\StringTok{\textasciigrave{}} \OtherTok{=} \FunctionTok{format}\NormalTok{(rddens}\SpecialCharTok{$}\NormalTok{N}\SpecialCharTok{$}\NormalTok{right, }\AttributeTok{big.mark =} \StringTok{","}\NormalTok{),}
  \StringTok{\textasciigrave{}}\AttributeTok{Eff. N (Left)}\StringTok{\textasciigrave{}} \OtherTok{=} \FunctionTok{format}\NormalTok{(rddens}\SpecialCharTok{$}\NormalTok{N}\SpecialCharTok{$}\NormalTok{eff\_left, }\AttributeTok{big.mark =} \StringTok{","}\NormalTok{),}
  \StringTok{\textasciigrave{}}\AttributeTok{Eff. N (Right)}\StringTok{\textasciigrave{}} \OtherTok{=} \FunctionTok{format}\NormalTok{(rddens}\SpecialCharTok{$}\NormalTok{N}\SpecialCharTok{$}\NormalTok{eff\_right, }\AttributeTok{big.mark =} \StringTok{","}\NormalTok{),}
  \AttributeTok{check.names =} \ConstantTok{FALSE}
\NormalTok{)}

\FunctionTok{kable}\NormalTok{(density\_results,}
      \AttributeTok{booktabs =} \ConstantTok{TRUE}\NormalTok{,}
      \AttributeTok{align =} \FunctionTok{c}\NormalTok{(}\StringTok{"l"}\NormalTok{, }\StringTok{"r"}\NormalTok{, }\StringTok{"r"}\NormalTok{, }\StringTok{"c"}\NormalTok{, }\StringTok{"r"}\NormalTok{, }\StringTok{"r"}\NormalTok{, }\StringTok{"r"}\NormalTok{, }\StringTok{"r"}\NormalTok{),}
      \AttributeTok{caption =} \StringTok{"Manipulation Test: Continuity of Density at Threshold"}\NormalTok{) }\SpecialCharTok{\%\textgreater{}\%}
  \FunctionTok{kable\_styling}\NormalTok{(}\AttributeTok{full\_width =} \ConstantTok{FALSE}\NormalTok{, }\AttributeTok{position =} \StringTok{"center"}\NormalTok{)}
\end{Highlighting}
\end{Shaded}

\begin{longtable}[t]{lrrcrrrr}
\caption{\label{tab:density-test}Manipulation Test: Continuity of Density at Threshold}\\
\toprule
Test & Statistic & p-value & Bandwidth & N (Left) & N (Right) & Eff. N (Left) & Eff. N (Right)\\
\midrule
Manipulation Test & -0.0757 & 0.9397 & 15.49 / 13.39 & 6,088 & 7,220 & 3,852 & 3,590\\
\bottomrule
\end{longtable}

\emph{Note: Test statistic based on local polynomial density estimation
with triangular kernel. Null hypothesis: No discontinuity in density at
threshold (no manipulation). High p-value indicates no evidence of
manipulation around the cutoff.}

\textbf{Interpretation:}

The density test checks whether there is evidence of manipulation around
the threshold (margin of victory = 0). A significant discontinuity in
the density would suggest that parties can manipulate their vote margins
to barely win elections, which would violate the RD identifying
assumptions.

\subsubsection{Covariate Balance Tests}\label{covariate-balance-tests}

\begin{Shaded}
\begin{Highlighting}[]
\CommentTok{\# Initialize lists to store results}
\NormalTok{balance\_results }\OtherTok{\textless{}{-}} \FunctionTok{list}\NormalTok{()}

\CommentTok{\# Test for balance in covariates at the threshold}
\ControlFlowTok{for}\NormalTok{(c }\ControlFlowTok{in} \DecValTok{1}\SpecialCharTok{:}\FunctionTok{ncol}\NormalTok{(covs))\{}
  \FunctionTok{cat}\NormalTok{(}\StringTok{"}\SpecialCharTok{\textbackslash{}n}\StringTok{"}\NormalTok{)}
\NormalTok{  rdr\_cov }\OtherTok{\textless{}{-}} \FunctionTok{rdrobust}\NormalTok{(covs[,c], X)}
\NormalTok{  balance\_results[[c]] }\OtherTok{\textless{}{-}}\NormalTok{ rdr\_cov}
  
  \CommentTok{\# Create RD plot for this covariate}
  \FunctionTok{rdplot}\NormalTok{(covs[,c], X,}
         \AttributeTok{y.label =}\NormalTok{ covsnm[c],}
         \AttributeTok{x.label =} \StringTok{"Incumbent Party\textquotesingle{}s Margin of Victory"}\NormalTok{,}
         \AttributeTok{x.lim =} \FunctionTok{c}\NormalTok{(}\SpecialCharTok{{-}}\DecValTok{30}\NormalTok{, }\DecValTok{30}\NormalTok{),}
         \AttributeTok{binselect =} \StringTok{"qsmv"}\NormalTok{)}
\NormalTok{\}}
\end{Highlighting}
\end{Shaded}

\pandocbounded{\includegraphics[keepaspectratio]{Tutorial-Week-11-IV-RDD-Analysis_files/figure-latex/covariate-balance-1.pdf}}
\pandocbounded{\includegraphics[keepaspectratio]{Tutorial-Week-11-IV-RDD-Analysis_files/figure-latex/covariate-balance-2.pdf}}
\pandocbounded{\includegraphics[keepaspectratio]{Tutorial-Week-11-IV-RDD-Analysis_files/figure-latex/covariate-balance-3.pdf}}
\pandocbounded{\includegraphics[keepaspectratio]{Tutorial-Week-11-IV-RDD-Analysis_files/figure-latex/covariate-balance-4.pdf}}
\pandocbounded{\includegraphics[keepaspectratio]{Tutorial-Week-11-IV-RDD-Analysis_files/figure-latex/covariate-balance-5.pdf}}
\pandocbounded{\includegraphics[keepaspectratio]{Tutorial-Week-11-IV-RDD-Analysis_files/figure-latex/covariate-balance-6.pdf}}
\pandocbounded{\includegraphics[keepaspectratio]{Tutorial-Week-11-IV-RDD-Analysis_files/figure-latex/covariate-balance-7.pdf}}

\begin{Shaded}
\begin{Highlighting}[]
\CommentTok{\# Create a summary table of balance tests}
\NormalTok{balance\_table }\OtherTok{\textless{}{-}} \FunctionTok{data.frame}\NormalTok{(}
  \AttributeTok{Covariate =}\NormalTok{ covsnm,}
  \AttributeTok{Coefficient =} \FunctionTok{sprintf}\NormalTok{(}\StringTok{"\%.4f"}\NormalTok{, }\FunctionTok{sapply}\NormalTok{(balance\_results, }\ControlFlowTok{function}\NormalTok{(x) x}\SpecialCharTok{$}\NormalTok{coef[}\DecValTok{1}\NormalTok{])),}
  \StringTok{\textasciigrave{}}\AttributeTok{Robust SE}\StringTok{\textasciigrave{}} \OtherTok{=} \FunctionTok{sprintf}\NormalTok{(}\StringTok{"\%.4f"}\NormalTok{, }\FunctionTok{sapply}\NormalTok{(balance\_results, }\ControlFlowTok{function}\NormalTok{(x) x}\SpecialCharTok{$}\NormalTok{se[}\DecValTok{3}\NormalTok{])),}
  \StringTok{\textasciigrave{}}\AttributeTok{t{-}statistic}\StringTok{\textasciigrave{}} \OtherTok{=} \FunctionTok{sprintf}\NormalTok{(}\StringTok{"\%.3f"}\NormalTok{, }\FunctionTok{sapply}\NormalTok{(balance\_results, }\ControlFlowTok{function}\NormalTok{(x) x}\SpecialCharTok{$}\NormalTok{z[}\DecValTok{3}\NormalTok{])),}
  \StringTok{\textasciigrave{}}\AttributeTok{p{-}value}\StringTok{\textasciigrave{}} \OtherTok{=} \FunctionTok{sprintf}\NormalTok{(}\StringTok{"\%.3f"}\NormalTok{, }\FunctionTok{sapply}\NormalTok{(balance\_results, }\ControlFlowTok{function}\NormalTok{(x) x}\SpecialCharTok{$}\NormalTok{pv[}\DecValTok{3}\NormalTok{])),}
  \StringTok{\textasciigrave{}}\AttributeTok{N (Left)}\StringTok{\textasciigrave{}} \OtherTok{=} \FunctionTok{format}\NormalTok{(}\FunctionTok{sapply}\NormalTok{(balance\_results, }\ControlFlowTok{function}\NormalTok{(x) x}\SpecialCharTok{$}\NormalTok{N\_h[}\DecValTok{1}\NormalTok{]), }\AttributeTok{big.mark =} \StringTok{","}\NormalTok{),}
  \StringTok{\textasciigrave{}}\AttributeTok{N (Right)}\StringTok{\textasciigrave{}} \OtherTok{=} \FunctionTok{format}\NormalTok{(}\FunctionTok{sapply}\NormalTok{(balance\_results, }\ControlFlowTok{function}\NormalTok{(x) x}\SpecialCharTok{$}\NormalTok{N\_h[}\DecValTok{2}\NormalTok{]), }\AttributeTok{big.mark =} \StringTok{","}\NormalTok{),}
  \AttributeTok{check.names =} \ConstantTok{FALSE}
\NormalTok{)}

\FunctionTok{kable}\NormalTok{(balance\_table,}
      \AttributeTok{booktabs =} \ConstantTok{TRUE}\NormalTok{,}
      \AttributeTok{align =} \FunctionTok{c}\NormalTok{(}\StringTok{"l"}\NormalTok{, }\StringTok{"r"}\NormalTok{, }\StringTok{"r"}\NormalTok{, }\StringTok{"r"}\NormalTok{, }\StringTok{"r"}\NormalTok{, }\StringTok{"r"}\NormalTok{, }\StringTok{"r"}\NormalTok{),}
      \AttributeTok{caption =} \StringTok{"Covariate Balance Tests at Threshold"}\NormalTok{) }\SpecialCharTok{\%\textgreater{}\%}
  \FunctionTok{kable\_styling}\NormalTok{(}\AttributeTok{full\_width =} \ConstantTok{FALSE}\NormalTok{, }\AttributeTok{position =} \StringTok{"center"}\NormalTok{)}
\end{Highlighting}
\end{Shaded}

\begin{longtable}[t]{lrrrrrr}
\caption{\label{tab:covariate-balance}Covariate Balance Tests at Threshold}\\
\toprule
Covariate & Coefficient & Robust SE & t-statistic & p-value & N (Left) & N (Right)\\
\midrule
GDP per capita & -149.9175 & 560.7565 & -0.190 & 0.849 & 3,626 & 3,740\\
Population & -22.5594 & 2124.3938 & 0.177 & 0.860 & 2,514 & 2,543\\
No. Effective Parties & -0.0363 & 0.0301 & -1.368 & 0.171 & 3,312 & 3,409\\
DEM Victory t-1 & 0.0049 & 0.0180 & 0.119 & 0.905 & 4,026 & 4,183\\
PSDB Victory t-1 & 0.0228 & 0.0199 & 1.439 & 0.150 & 3,800 & 3,914\\
\addlinespace
PT Victory t-1 & 0.0131 & 0.0135 & 1.000 & 0.317 & 4,407 & 4,645\\
PMDB Victory t-1 & -0.0347 & 0.0234 & -1.707 & 0.088 & 3,837 & 3,962\\
\bottomrule
\end{longtable}

\emph{Note: Robust bias-corrected RD estimates using MSE-optimal
bandwidth selection. Standard errors calculated using nearest-neighbor
variance estimator.}

\textbf{Interpretation:}

Covariate balance tests check whether pre-treatment characteristics are
continuous at the threshold. If covariates jump discontinuously at the
cutoff, this suggests the treatment is not ``as-if'' randomly assigned,
which would undermine the validity of the RD design.

\subsection{Part C - Outcome Analysis}\label{part-c---outcome-analysis}

\subsubsection{RD Plot}\label{rd-plot}

\begin{Shaded}
\begin{Highlighting}[]
\CommentTok{\# Create RD plot of outcome variable}
\FunctionTok{rdplot}\NormalTok{(Y, X,}
       \AttributeTok{y.label =} \StringTok{"Incumbent Party Victory at t+1"}\NormalTok{,}
       \AttributeTok{x.label =} \StringTok{"Incumbent Party\textquotesingle{}s Margin of Victory at t"}\NormalTok{)}
\end{Highlighting}
\end{Shaded}

\pandocbounded{\includegraphics[keepaspectratio]{Tutorial-Week-11-IV-RDD-Analysis_files/figure-latex/rd-plot-1.pdf}}

\subsubsection{Continuity-Based RDD
Analysis}\label{continuity-based-rdd-analysis}

\begin{Shaded}
\begin{Highlighting}[]
\CommentTok{\# Main RDD estimate without covariates}
\NormalTok{rdr }\OtherTok{\textless{}{-}} \FunctionTok{rdrobust}\NormalTok{(Y, X)}

\CommentTok{\# RDD estimate with covariates}
\NormalTok{rdrcovs }\OtherTok{\textless{}{-}} \FunctionTok{rdrobust}\NormalTok{(Y, X, }\AttributeTok{covs =}\NormalTok{ covs)}

\CommentTok{\# Create a comprehensive results table}
\NormalTok{results\_table }\OtherTok{\textless{}{-}} \FunctionTok{data.frame}\NormalTok{(}
  \AttributeTok{Specification =} \FunctionTok{c}\NormalTok{(}\StringTok{"Without Covariates"}\NormalTok{, }\StringTok{"With Covariates"}\NormalTok{),}
  \AttributeTok{Coefficient =} \FunctionTok{sprintf}\NormalTok{(}\StringTok{"\%.4f"}\NormalTok{, }\FunctionTok{c}\NormalTok{(rdr}\SpecialCharTok{$}\NormalTok{coef[}\DecValTok{1}\NormalTok{], rdrcovs}\SpecialCharTok{$}\NormalTok{coef[}\DecValTok{1}\NormalTok{])),}
  \StringTok{\textasciigrave{}}\AttributeTok{Conv. SE}\StringTok{\textasciigrave{}} \OtherTok{=} \FunctionTok{sprintf}\NormalTok{(}\StringTok{"\%.4f"}\NormalTok{, }\FunctionTok{c}\NormalTok{(rdr}\SpecialCharTok{$}\NormalTok{se[}\DecValTok{1}\NormalTok{], rdrcovs}\SpecialCharTok{$}\NormalTok{se[}\DecValTok{1}\NormalTok{])),}
  \StringTok{\textasciigrave{}}\AttributeTok{Robust SE}\StringTok{\textasciigrave{}} \OtherTok{=} \FunctionTok{sprintf}\NormalTok{(}\StringTok{"\%.4f"}\NormalTok{, }\FunctionTok{c}\NormalTok{(rdr}\SpecialCharTok{$}\NormalTok{se[}\DecValTok{3}\NormalTok{], rdrcovs}\SpecialCharTok{$}\NormalTok{se[}\DecValTok{3}\NormalTok{])),}
  \StringTok{\textasciigrave{}}\AttributeTok{p{-}value}\StringTok{\textasciigrave{}} \OtherTok{=} \FunctionTok{sprintf}\NormalTok{(}\StringTok{"\%.3f"}\NormalTok{, }\FunctionTok{c}\NormalTok{(rdr}\SpecialCharTok{$}\NormalTok{pv[}\DecValTok{3}\NormalTok{], rdrcovs}\SpecialCharTok{$}\NormalTok{pv[}\DecValTok{3}\NormalTok{])),}
  \StringTok{\textasciigrave{}}\AttributeTok{95\% CI}\StringTok{\textasciigrave{}} \OtherTok{=} \FunctionTok{sprintf}\NormalTok{(}\StringTok{"[\%.3f, \%.3f]"}\NormalTok{, }\FunctionTok{c}\NormalTok{(rdr}\SpecialCharTok{$}\NormalTok{ci[}\DecValTok{3}\NormalTok{,}\DecValTok{1}\NormalTok{], rdrcovs}\SpecialCharTok{$}\NormalTok{ci[}\DecValTok{3}\NormalTok{,}\DecValTok{1}\NormalTok{]), }
                                      \FunctionTok{c}\NormalTok{(rdr}\SpecialCharTok{$}\NormalTok{ci[}\DecValTok{3}\NormalTok{,}\DecValTok{2}\NormalTok{], rdrcovs}\SpecialCharTok{$}\NormalTok{ci[}\DecValTok{3}\NormalTok{,}\DecValTok{2}\NormalTok{])),}
  \StringTok{\textasciigrave{}}\AttributeTok{BW (L/R)}\StringTok{\textasciigrave{}} \OtherTok{=} \FunctionTok{sprintf}\NormalTok{(}\StringTok{"\%.2f / \%.2f"}\NormalTok{, }\FunctionTok{c}\NormalTok{(rdr}\SpecialCharTok{$}\NormalTok{bws[}\DecValTok{1}\NormalTok{], rdrcovs}\SpecialCharTok{$}\NormalTok{bws[}\DecValTok{1}\NormalTok{]), }
                                      \FunctionTok{c}\NormalTok{(rdr}\SpecialCharTok{$}\NormalTok{bws[}\DecValTok{2}\NormalTok{], rdrcovs}\SpecialCharTok{$}\NormalTok{bws[}\DecValTok{2}\NormalTok{])),}
  \StringTok{\textasciigrave{}}\AttributeTok{N (Left)}\StringTok{\textasciigrave{}} \OtherTok{=} \FunctionTok{format}\NormalTok{(}\FunctionTok{c}\NormalTok{(rdr}\SpecialCharTok{$}\NormalTok{N\_h[}\DecValTok{1}\NormalTok{], rdrcovs}\SpecialCharTok{$}\NormalTok{N\_h[}\DecValTok{1}\NormalTok{]), }\AttributeTok{big.mark =} \StringTok{","}\NormalTok{),}
  \StringTok{\textasciigrave{}}\AttributeTok{N (Right)}\StringTok{\textasciigrave{}} \OtherTok{=} \FunctionTok{format}\NormalTok{(}\FunctionTok{c}\NormalTok{(rdr}\SpecialCharTok{$}\NormalTok{N\_h[}\DecValTok{2}\NormalTok{], rdrcovs}\SpecialCharTok{$}\NormalTok{N\_h[}\DecValTok{2}\NormalTok{]), }\AttributeTok{big.mark =} \StringTok{","}\NormalTok{),}
  \AttributeTok{check.names =} \ConstantTok{FALSE}
\NormalTok{)}

\FunctionTok{kable}\NormalTok{(results\_table,}
      \AttributeTok{booktabs =} \ConstantTok{TRUE}\NormalTok{,}
      \AttributeTok{align =} \FunctionTok{c}\NormalTok{(}\StringTok{"l"}\NormalTok{, }\StringTok{"r"}\NormalTok{, }\StringTok{"r"}\NormalTok{, }\StringTok{"r"}\NormalTok{, }\StringTok{"r"}\NormalTok{, }\StringTok{"c"}\NormalTok{, }\StringTok{"c"}\NormalTok{, }\StringTok{"r"}\NormalTok{, }\StringTok{"r"}\NormalTok{),}
      \AttributeTok{caption =} \StringTok{"Regression Discontinuity Estimates: Effect of Incumbent Party Victory on Future Electoral Success"}\NormalTok{) }\SpecialCharTok{\%\textgreater{}\%}
  \FunctionTok{kable\_styling}\NormalTok{(}\AttributeTok{full\_width =} \ConstantTok{FALSE}\NormalTok{, }\AttributeTok{position =} \StringTok{"center"}\NormalTok{)}
\end{Highlighting}
\end{Shaded}

\begin{longtable}[t]{lrrrrccrr}
\caption{\label{tab:rd-analysis-main}Regression Discontinuity Estimates: Effect of Incumbent Party Victory on Future Electoral Success}\\
\toprule
Specification & Coefficient & Conv. SE & Robust SE & p-value & 95\% CI & BW (L/R) & N (Left) & N (Right)\\
\midrule
Without Covariates & -6.2813 & 1.6008 & 1.8571 & 0.000 & {}[-10.223, -2.944] & 15.29 / 27.51 & 1,533 & 1,740\\
With Covariates & -6.1061 & 1.5782 & 1.8433 & 0.001 & {}[-9.881, -2.655] & 14.45 / 25.25 & 1,481 & 1,672\\
\bottomrule
\end{longtable}

\begin{Shaded}
\begin{Highlighting}[]
\CommentTok{\# Print detailed summary for reference}
\FunctionTok{cat}\NormalTok{(}\StringTok{"}\SpecialCharTok{\textbackslash{}n}\StringTok{=== DETAILED RDD ESTIMATES ===}\SpecialCharTok{\textbackslash{}n\textbackslash{}n}\StringTok{"}\NormalTok{)}
\end{Highlighting}
\end{Shaded}

\begin{verbatim}
## 
## === DETAILED RDD ESTIMATES ===
\end{verbatim}

\begin{Shaded}
\begin{Highlighting}[]
\FunctionTok{cat}\NormalTok{(}\StringTok{"{-}{-}{-} Without Covariates {-}{-}{-}}\SpecialCharTok{\textbackslash{}n}\StringTok{"}\NormalTok{)}
\end{Highlighting}
\end{Shaded}

\begin{verbatim}
## --- Without Covariates ---
\end{verbatim}

\begin{Shaded}
\begin{Highlighting}[]
\FunctionTok{print}\NormalTok{(}\FunctionTok{summary}\NormalTok{(rdr))}
\end{Highlighting}
\end{Shaded}

\begin{verbatim}
## Sharp RD estimates using local polynomial regression.
## 
## Number of Obs.                 5463
## BW type                       mserd
## Kernel                   Triangular
## VCE method                       NN
## 
## Number of Obs.                 2220         3243
## Eff. Number of Obs.            1533         1740
## Order est. (p)                    1            1
## Order bias  (q)                   2            2
## BW est. (h)                  15.291       15.291
## BW bias (b)                  27.509       27.509
## rho (h/b)                     0.556        0.556
## Unique Obs.                    2213         3119
## 
## =====================================================================
##                    Point    Robust Inference
##                 Estimate         z     P>|z|      [ 95% C.I. ]       
## ---------------------------------------------------------------------
##      RD Effect    -6.281    -3.545     0.000   [-10.223 , -2.944]    
## =====================================================================
## NULL
\end{verbatim}

\begin{Shaded}
\begin{Highlighting}[]
\FunctionTok{cat}\NormalTok{(}\StringTok{"}\SpecialCharTok{\textbackslash{}n}\StringTok{{-}{-}{-} With Covariates {-}{-}{-}}\SpecialCharTok{\textbackslash{}n}\StringTok{"}\NormalTok{)}
\end{Highlighting}
\end{Shaded}

\begin{verbatim}
## 
## --- With Covariates ---
\end{verbatim}

\begin{Shaded}
\begin{Highlighting}[]
\FunctionTok{print}\NormalTok{(}\FunctionTok{summary}\NormalTok{(rdrcovs))}
\end{Highlighting}
\end{Shaded}

\begin{verbatim}
## Covariate-adjusted Sharp RD estimates using local polynomial regression.
## 
## Number of Obs.                 5460
## BW type                       mserd
## Kernel                   Triangular
## VCE method                       NN
## 
## Number of Obs.                 2218         3242
## Eff. Number of Obs.            1481         1672
## Order est. (p)                    1            1
## Order bias  (q)                   2            2
## BW est. (h)                  14.451       14.451
## BW bias (b)                  25.248       25.248
## rho (h/b)                     0.572        0.572
## Unique Obs.                    2211         3118
## 
## =====================================================================
##                    Point    Robust Inference
##                 Estimate         z     P>|z|      [ 95% C.I. ]       
## ---------------------------------------------------------------------
##      RD Effect    -6.106    -3.401     0.001    [-9.881 , -2.655]    
## =====================================================================
## NULL
\end{verbatim}

\emph{Notes: Dependent variable is Incumbent Party Victory at t+1 (in
percentage points). Robust bias-corrected confidence intervals and
p-values reported. MSE-optimal bandwidth selection with triangular
kernel. BW (L/R) shows left/right bandwidths. Covariates include GDP per
capita, Population, No.~Effective Parties, and DEM/PSDB/PT/PMDB Victory
at t-1.}

\textbf{Interpretation:}

The RDD estimates show the causal effect of barely winning an election
(vs.~barely losing) on the probability of the incumbent party winning
the next election. This tests the ``incumbency curse'' hypothesis - that
winning may actually hurt a party's chances in the next election due to
weak parties and term limits in Brazilian municipalities.

Compare the estimates with and without covariates. If they are similar,
this provides additional evidence that the RD design is valid
(covariates should not matter much if treatment is as-if random near the
threshold).

\subsection{Part D - Local Randomization
Approach}\label{part-d---local-randomization-approach}

\begin{Shaded}
\begin{Highlighting}[]
\CommentTok{\# Window selection for local randomization}
\NormalTok{rdwin }\OtherTok{\textless{}{-}} \FunctionTok{rdwinselect}\NormalTok{(X, covs, }\AttributeTok{wmin =} \FloatTok{0.05}\NormalTok{, }\AttributeTok{wstep =} \FloatTok{0.01}\NormalTok{, }\AttributeTok{nwindows =} \DecValTok{200}\NormalTok{,}
                     \AttributeTok{seed =} \DecValTok{765}\NormalTok{, }\AttributeTok{plot =} \ConstantTok{TRUE}\NormalTok{, }\AttributeTok{quietly =} \ConstantTok{TRUE}\NormalTok{)}
\end{Highlighting}
\end{Shaded}

\begin{verbatim}
## Mass points detected in running variable
## You may use wmasspoints option for constructing windows at each mass point
\end{verbatim}

\pandocbounded{\includegraphics[keepaspectratio]{Tutorial-Week-11-IV-RDD-Analysis_files/figure-latex/rd-local-random-1.pdf}}

\begin{Shaded}
\begin{Highlighting}[]
\CommentTok{\# Use selected window (or manually choose)}
\NormalTok{w }\OtherTok{\textless{}{-}} \FloatTok{0.15}

\CommentTok{\# Randomization inference}
\NormalTok{rdrand }\OtherTok{\textless{}{-}} \FunctionTok{rdrandinf}\NormalTok{(Y, X, }\AttributeTok{wl =} \SpecialCharTok{{-}}\NormalTok{w, }\AttributeTok{wr =}\NormalTok{ w, }\AttributeTok{reps =} \DecValTok{1000}\NormalTok{, }\AttributeTok{seed =} \DecValTok{765}\NormalTok{)}
\end{Highlighting}
\end{Shaded}

\begin{verbatim}
## 
## Selected window = [-0.15;0.15] 
## 
## Running randomization-based test...
## Randomization-based test complete. 
## 
## 
## Number of obs     =          5463
## Order of poly     =             0
## Kernel type       =       uniform
## Reps              =          1000
## Window            =   set by user
## H0:          tau  =         0.000
## Randomization     = fixed margins
## 
## Cutoff c =    0.000   Left of c  Right of c
##       Number of obs        2220        3243
##  Eff. number of obs          19          20
##     Mean of outcome       0.631      -9.361
##     S.d. of outcome      17.733      16.610
##              Window      -0.150       0.150
## 
## ================================================================================
##                                   Finite sample            Large sample         
##                                ------------------  -----------------------------
##           Statistic          T        P>|T|        P>|T|    Power vs d =   8.867
## ================================================================================
##      Diff. in means     -9.992        0.076        0.070                   0.363
## ================================================================================
\end{verbatim}

\begin{Shaded}
\begin{Highlighting}[]
\CommentTok{\# Create local randomization results table}
\NormalTok{local\_rand\_table }\OtherTok{\textless{}{-}} \FunctionTok{data.frame}\NormalTok{(}
  \AttributeTok{Approach =} \FunctionTok{c}\NormalTok{(}\StringTok{"Continuity{-}Based"}\NormalTok{, }\StringTok{"Local Randomization"}\NormalTok{),}
  \AttributeTok{Window =} \FunctionTok{c}\NormalTok{(}\FunctionTok{sprintf}\NormalTok{(}\StringTok{"±\%.2f / ±\%.2f"}\NormalTok{, rdr}\SpecialCharTok{$}\NormalTok{bws[}\DecValTok{1}\NormalTok{], rdr}\SpecialCharTok{$}\NormalTok{bws[}\DecValTok{2}\NormalTok{]),}
             \FunctionTok{sprintf}\NormalTok{(}\StringTok{"±\%.2f"}\NormalTok{, w)),}
  \AttributeTok{Coefficient =} \FunctionTok{sprintf}\NormalTok{(}\StringTok{"\%.4f"}\NormalTok{, }\FunctionTok{c}\NormalTok{(rdr}\SpecialCharTok{$}\NormalTok{coef[}\DecValTok{1}\NormalTok{], rdrand}\SpecialCharTok{$}\NormalTok{obs.stat)),}
  \StringTok{\textasciigrave{}}\AttributeTok{Robust SE}\StringTok{\textasciigrave{}} \OtherTok{=} \FunctionTok{c}\NormalTok{(}\FunctionTok{sprintf}\NormalTok{(}\StringTok{"\%.4f"}\NormalTok{, rdr}\SpecialCharTok{$}\NormalTok{se[}\DecValTok{3}\NormalTok{]), }\StringTok{"—"}\NormalTok{),}
  \StringTok{\textasciigrave{}}\AttributeTok{p{-}value}\StringTok{\textasciigrave{}} \OtherTok{=} \FunctionTok{sprintf}\NormalTok{(}\StringTok{"\%.3f"}\NormalTok{, }\FunctionTok{c}\NormalTok{(rdr}\SpecialCharTok{$}\NormalTok{pv[}\DecValTok{3}\NormalTok{], rdrand}\SpecialCharTok{$}\NormalTok{p.value)),}
  \StringTok{\textasciigrave{}}\AttributeTok{95\% CI}\StringTok{\textasciigrave{}} \OtherTok{=} \FunctionTok{c}\NormalTok{(}\FunctionTok{sprintf}\NormalTok{(}\StringTok{"[\%.3f, \%.3f]"}\NormalTok{, rdr}\SpecialCharTok{$}\NormalTok{ci[}\DecValTok{3}\NormalTok{,}\DecValTok{1}\NormalTok{], rdr}\SpecialCharTok{$}\NormalTok{ci[}\DecValTok{3}\NormalTok{,}\DecValTok{2}\NormalTok{]), }\StringTok{"—"}\NormalTok{),}
  \StringTok{\textasciigrave{}}\AttributeTok{N (Left)}\StringTok{\textasciigrave{}} \OtherTok{=} \FunctionTok{format}\NormalTok{(}\FunctionTok{c}\NormalTok{(rdr}\SpecialCharTok{$}\NormalTok{N\_h[}\DecValTok{1}\NormalTok{], rdrand}\SpecialCharTok{$}\NormalTok{sumstats[}\DecValTok{1}\NormalTok{]), }\AttributeTok{big.mark =} \StringTok{","}\NormalTok{),}
  \StringTok{\textasciigrave{}}\AttributeTok{N (Right)}\StringTok{\textasciigrave{}} \OtherTok{=} \FunctionTok{format}\NormalTok{(}\FunctionTok{c}\NormalTok{(rdr}\SpecialCharTok{$}\NormalTok{N\_h[}\DecValTok{2}\NormalTok{], rdrand}\SpecialCharTok{$}\NormalTok{sumstats[}\DecValTok{2}\NormalTok{]), }\AttributeTok{big.mark =} \StringTok{","}\NormalTok{),}
  \AttributeTok{Method =} \FunctionTok{c}\NormalTok{(}\StringTok{"Asymptotic"}\NormalTok{, }\StringTok{"Permutation"}\NormalTok{),}
  \AttributeTok{check.names =} \ConstantTok{FALSE}
\NormalTok{)}

\FunctionTok{kable}\NormalTok{(local\_rand\_table,}
      \AttributeTok{booktabs =} \ConstantTok{TRUE}\NormalTok{,}
      \AttributeTok{align =} \FunctionTok{c}\NormalTok{(}\StringTok{"l"}\NormalTok{, }\StringTok{"c"}\NormalTok{, }\StringTok{"r"}\NormalTok{, }\StringTok{"r"}\NormalTok{, }\StringTok{"r"}\NormalTok{, }\StringTok{"c"}\NormalTok{, }\StringTok{"r"}\NormalTok{, }\StringTok{"r"}\NormalTok{, }\StringTok{"l"}\NormalTok{),}
      \AttributeTok{caption =} \StringTok{"Comparison of RDD Approaches: Continuity{-}Based vs. Local Randomization"}\NormalTok{) }\SpecialCharTok{\%\textgreater{}\%}
  \FunctionTok{kable\_styling}\NormalTok{(}\AttributeTok{full\_width =} \ConstantTok{FALSE}\NormalTok{, }\AttributeTok{position =} \StringTok{"center"}\NormalTok{)}
\end{Highlighting}
\end{Shaded}

\begin{longtable}[t]{lcrrrcrrl}
\caption{\label{tab:rd-local-random}Comparison of RDD Approaches: Continuity-Based vs. Local Randomization}\\
\toprule
Approach & Window & Coefficient & Robust SE & p-value & 95\% CI & N (Left) & N (Right) & Method\\
\midrule
Continuity-Based & ±15.29 / ±27.51 & -6.2813 & 1.8571 & 0.000 & {}[-10.223, -2.944] & 1,533 & 1,740 & Asymptotic\\
Local Randomization & ±0.15 & -9.9923 & — & 0.076 & — & 2,220 & 19 & Permutation\\
\bottomrule
\end{longtable}

\begin{Shaded}
\begin{Highlighting}[]
\FunctionTok{cat}\NormalTok{(}\StringTok{"}\SpecialCharTok{\textbackslash{}n}\StringTok{=== LOCAL RANDOMIZATION INFERENCE (Detailed) ===}\SpecialCharTok{\textbackslash{}n}\StringTok{"}\NormalTok{)}
\end{Highlighting}
\end{Shaded}

\begin{verbatim}
## 
## === LOCAL RANDOMIZATION INFERENCE (Detailed) ===
\end{verbatim}

\begin{Shaded}
\begin{Highlighting}[]
\FunctionTok{print}\NormalTok{(}\FunctionTok{summary}\NormalTok{(rdrand))}
\end{Highlighting}
\end{Shaded}

\begin{verbatim}
##            Length Class  Mode   
## sumstats   10     -none- numeric
## obs.stat    1     -none- numeric
## p.value     1     -none- numeric
## asy.pvalue  1     -none- numeric
## window      2     -none- numeric
\end{verbatim}

\emph{Notes: Dependent variable is Incumbent Party Victory at t+1 (in
percentage points). Local randomization uses 1,000 permutations with
fixed margins assumption. Continuity-based approach uses MSE-optimal
bandwidth with robust bias-correction. Local randomization assumes as-if
random assignment within the narrow window.}

\textbf{Interpretation:}

The local randomization approach assumes that units very close to the
threshold (within a narrow window) are essentially randomly assigned to
treatment. This provides an alternative inference method that doesn't
rely on asymptotic approximations and may be more appropriate with
discrete running variables.

\subsection{Part E - Assessment}\label{part-e---assessment}

\textbf{Key Findings:}

\begin{enumerate}
\def\labelenumi{\arabic{enumi}.}
\item
  \textbf{Density Test:} {[}Interpret whether there is evidence of
  manipulation{]}
\item
  \textbf{Covariate Balance:} {[}Summarize whether covariates are
  balanced{]}
\item
  \textbf{RDD Estimates:} {[}State the main findings about the
  incumbency effect{]}
\item
  \textbf{Robustness:} {[}Assess whether estimates are stable across
  specifications{]}
\end{enumerate}

\textbf{Overall Validity:}

{[}Your assessment of whether the RDD design is credible in this context
and whether the findings support the ``incumbency curse'' hypothesis{]}

\end{document}
